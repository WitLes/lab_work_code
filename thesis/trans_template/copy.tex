
%% bare_jrnl.tex
%% V1.2
%% 2002/11/18
%% by Michael Shell
%% mshell@ece.gatech.edu
%%
%% NOTE: This text file uses MS Windows line feed conventions. When (human)
%% reading this file on other platforms, you may have to use a text
%% editor that can handle lines terminated by the MS Windows line feed
%% characters (0x0D 0x0A).
%%
%% This is a skeleton file demonstrating the use of IEEEtran.cls
%% (requires IEEEtran.cls version 1.6b or later) with an IEEE journal paper.
%%
%% Support sites:
%% http://www.ieee.org
%% and/or
%% http://www.ctan.org/tex-archive/macros/latex/contrib/supported/IEEEtran/
%%
%% This code is offered as-is - no warranty - user assumes all risk.
%% Free to use, distribute and modify.

% *** Authors should verify (and, if needed, correct) their LaTeX system  ***
% *** with the testflow diagnostic prior to trusting their LaTeX platform ***
% *** with production work. IEEE's font choices can trigger bugs that do  ***
% *** not appear when using other class files.                            ***
% Testflow can be obtained at:
% http://www.ctan.org/tex-archive/macros/latex/contrib/supported/IEEEtran/testflow


% Note that the a4paper option is mainly intended so that authors in
% countries using A4 can easily print to A4 and see how their papers will
% look in print. Authors are encouraged to use U.S. letter paper when 
% submitting to IEEE. Use the testflow package mentioned above to verify
% correct handling of both paper sizes by the author's LaTeX system.
%
% Also note that the "draftcls" or "draftclsnofoot", not "draft", option
% should be used if it is desired that the figures are to be displayed in
% draft mode.
%
% This example can be formatted using the peerreview
% (instead of journal) mode.
\documentclass[journal]{IEEEtran}
% If the IEEEtran.cls has not been installed into the LaTeX system files,
% manually specify the path to it:
% \documentclass[journal]{../sty/IEEEtran}

% some very useful LaTeX packages include:

%\usepackage{cite}      % Written by Donald Arseneau
% V1.6 and later of IEEEtran pre-defines the format
% of the cite.sty package \cite{} output to follow
% that of IEEE. Loading the cite package will
% result in citation numbers being automatically
% sorted and properly "ranged". i.e.,
% [1], [9], [2], [7], [5], [6]
% (without using cite.sty)
% will become:
% [1], [2], [5]--[7], [9] (using cite.sty)
% cite.sty's \cite will automatically add leading
% space, if needed. Use cite.sty's noadjust option
% (cite.sty V3.8 and later) if you want to turn this
% off. cite.sty is already installed on most LaTeX
% systems. The latest version can be obtained at:
% http://www.ctan.org/tex-archive/macros/latex/contrib/supported/cite/

\usepackage{graphicx}  % Written by David Carlisle and Sebastian Rahtz
\usepackage{booktabs}
\usepackage{ctex}
\usepackage{bm}
\usepackage{epstopdf}
\makeatletter  
\newif\if@restonecol  
\makeatother  
\let\algorithm\relax  
\let\endalgorithm\relax  
\makeatletter  
\newif\if@restonecol  
\makeatother  
\let\algorithm\relax  
\let\endalgorithm\relax  
\usepackage[linesnumbered,ruled,vlined]{algorithm2e}%[ruled,vlined]{  
\usepackage{algpseudocode}  
\usepackage{amsmath}  
\renewcommand{\algorithmicrequire}{\textbf{Input:}}  % Use Input in the format of Algorithm  
\renewcommand{\algorithmicensure}{\textbf{Output:}} % Use Output in the format of Algorithm   

% Required if you want graphics, photos, etc.
% graphicx.sty is already installed on most LaTeX
% systems. The latest version and documentation can
% be obtained at:
% http://www.ctan.org/tex-archive/macros/latex/required/graphics/
% Another good source of documentation is "Using
% Imported Graphics in LaTeX2e" by Keith Reckdahl
% which can be found as esplatex.ps and epslatex.pdf
% at: http://www.ctan.org/tex-archive/info/
% NOTE: for dual use with latex and pdflatex, instead load graphicx like:
%\ifx\pdfoutput\undefined
%\usepackage{graphicx}
%\else
%\usepackage[pdftex]{graphicx}
%\fi

% However, be warned that pdflatex will require graphics to be in PDF
% (not EPS) format and will preclude the use of PostScript based LaTeX
% packages such as psfrag.sty and pstricks.sty. IEEE conferences typically
% allow PDF graphics (and hence pdfLaTeX). However, IEEE journals do not
% (yet) allow image formats other than EPS or TIFF. Therefore, authors of
% journal papers should use traditional LaTeX with EPS graphics.
%
% The path(s) to the graphics files can also be declared: e.g.,
% \graphicspath{{../eps/}{../ps/}}
% if the graphics files are not located in the same directory as the
% .tex file. This can be done in each branch of the conditional above
% (after graphicx is loaded) to handle the EPS and PDF cases separately.
% In this way, full path information will not have to be specified in
% each \includegraphics command.
%
% Note that, when switching from latex to pdflatex and vice-versa, the new
% compiler will have to be run twice to clear some warnings.


%\usepackage{psfrag}    % Written by Craig Barratt, Michael C. Grant,
% and David Carlisle
% This package allows you to substitute LaTeX
% commands for text in imported EPS graphic files.
% In this way, LaTeX symbols can be placed into
% graphics that have been generated by other
% applications. You must use latex->dvips->ps2pdf
% workflow (not direct pdf output from pdflatex) if
% you wish to use this capability because it works
% via some PostScript tricks. Alternatively, the
% graphics could be processed as separate files via
% psfrag and dvips, then converted to PDF for
% inclusion in the main file which uses pdflatex.
% Docs are in "The PSfrag System" by Michael C. Grant
% and David Carlisle. There is also some information 
% about using psfrag in "Using Imported Graphics in
% LaTeX2e" by Keith Reckdahl which documents the
% graphicx package (see above). The psfrag package
% and documentation can be obtained at:
% http://www.ctan.org/tex-archive/macros/latex/contrib/supported/psfrag/

%\usepackage{subfigure} % Written by Steven Douglas Cochran
% This package makes it easy to put subfigures
% in your figures. i.e., "figure 1a and 1b"
% Docs are in "Using Imported Graphics in LaTeX2e"
% by Keith Reckdahl which also documents the graphicx
% package (see above). subfigure.sty is already
% installed on most LaTeX systems. The latest version
% and documentation can be obtained at:
% http://www.ctan.org/tex-archive/macros/latex/contrib/supported/subfigure/

%\usepackage{url}       % Written by Donald Arseneau
% Provides better support for handling and breaking
% URLs. url.sty is already installed on most LaTeX
% systems. The latest version can be obtained at:
% http://www.ctan.org/tex-archive/macros/latex/contrib/other/misc/
% Read the url.sty source comments for usage information.

%\usepackage{stfloats}  % Written by Sigitas Tolusis
% Gives LaTeX2e the ability to do double column
% floats at the bottom of the page as well as the top.
% (e.g., "\begin{figure*}[!b]" is not normally
% possible in LaTeX2e). This is an invasive package
% which rewrites many portions of the LaTeX2e output
% routines. It may not work with other packages that
% modify the LaTeX2e output routine and/or with other
% versions of LaTeX. The latest version and
% documentation can be obtained at:
% http://www.ctan.org/tex-archive/macros/latex/contrib/supported/sttools/
% Documentation is contained in the stfloats.sty
% comments as well as in the presfull.pdf file.
% Do not use the stfloats baselinefloat ability as
% IEEE does not allow \baselineskip to stretch.
% Authors submitting work to the IEEE should note
% that IEEE rarely uses double column equations and
% that authors should try to avoid such use.
% Do not be tempted to use the cuted.sty or
% midfloat.sty package (by the same author) as IEEE
% does not format its papers in such ways.

%\usepackage{amsmath}   % From the American Mathematical Society
% A popular package that provides many helpful commands
% for dealing with mathematics. Note that the AMSmath
% package sets \interdisplaylinepenalty to 10000 thus
% preventing page breaks from occurring within multiline
% equations. Use:
%\interdisplaylinepenalty=2500
% after loading amsmath to restore such page breaks
% as IEEEtran.cls normally does. amsmath.sty is already
% installed on most LaTeX systems. The latest version
% and documentation can be obtained at:
% http://www.ctan.org/tex-archive/macros/latex/required/amslatex/math/



% Other popular packages for formatting tables and equations include:

%\usepackage{array}
% Frank Mittelbach's and David Carlisle's array.sty which improves the
% LaTeX2e array and tabular environments to provide better appearances and
% additional user controls. array.sty is already installed on most systems.
% The latest version and documentation can be obtained at:
% http://www.ctan.org/tex-archive/macros/latex/required/tools/

% Mark Wooding's extremely powerful MDW tools, especially mdwmath.sty and
% mdwtab.sty which are used to format equations and tables, respectively.
% The MDWtools set is already installed on most LaTeX systems. The lastest
% version and documentation is available at:
% http://www.ctan.org/tex-archive/macros/latex/contrib/supported/mdwtools/


% V1.6 of IEEEtran contains the IEEEeqnarray family of commands that can
% be used to generate multiline equations as well as matrices, tables, etc.


% Also of notable interest:

% Scott Pakin's eqparbox package for creating (automatically sized) equal
% width boxes. Available:
% http://www.ctan.org/tex-archive/macros/latex/contrib/supported/eqparbox/



% Notes on hyperref:
% IEEEtran.cls attempts to be compliant with the hyperref package, written
% by Heiko Oberdiek and Sebastian Rahtz, which provides hyperlinks within
% a document as well as an index for PDF files (produced via pdflatex).
% However, it is a tad difficult to properly interface LaTeX classes and
% packages with this (necessarily) complex and invasive package. It is
% recommended that hyperref not be used for work that is to be submitted
% to the IEEE. Users who wish to use hyperref *must* ensure that their
% hyperref version is 6.72u or later *and* IEEEtran.cls is version 1.6b
% or later. The latest version of hyperref can be obtained at:
%
% http://www.ctan.org/tex-archive/macros/latex/contrib/supported/hyperref/
%
% Also, be aware that cite.sty (as of version 3.9, 11/2001) and hyperref.sty
% (as of version 6.72t, 2002/07/25) do not work optimally together.
% To mediate the differences between these two packages, IEEEtran.cls, as
% of v1.6b, predefines a command that fools hyperref into thinking that
% the natbib package is being used - causing it not to modify the existing
% citation commands, and allowing cite.sty to operate as normal. However,
% as a result, citation numbers will not be hyperlinked. Another side effect
% of this approach is that the natbib.sty package will not properly load
% under IEEEtran.cls. However, current versions of natbib are not capable
% of compressing and sorting citation numbers in IEEE's style - so this
% should not be an issue. If, for some strange reason, the user wants to
% load natbib.sty under IEEEtran.cls, the following code must be placed
% before natbib.sty can be loaded:
%
% \makeatletter
% \let\NAT@parse\undefined
% \makeatother
%
% Hyperref should be loaded differently depending on whether pdflatex
% or traditional latex is being used:
%
%\ifx\pdfoutput\undefined
%\usepackage[hypertex]{hyperref}
%\else
%\usepackage[pdftex,hypertexnames=false]{hyperref}
%\fi
%
% Pdflatex produces superior hyperref results and is the recommended
% compiler for such use.



% *** Do not adjust lengths that control margins, column widths, etc. ***
% *** Do not use packages that alter fonts (such as pslatex).         ***
% There should be no need to do such things with IEEEtran.cls V1.6 and later.


% correct bad hyphenation here
\hyphenation{op-tical net-works semi-conduc-tor}


\begin{document}
	%
	% paper title
	\title{Network Reconstruction through diffusive arrival times}
	%
	%
	% author names and IEEE memberships
	% note positions of commas and nonbreaking spaces ( ~ ) LaTeX will not break
	% a structure at a ~ so this keeps an author's name from being broken across
	% two lines.
	% use \thanks{} to gain access to the first footnote area
	% a separate \thanks must be used for each paragraph as LaTeX2e's \thanks
	% was not built to handle multiple paragraphs
	\author{}
	% <-this % stops a space}
	% note the % following the last \IEEEmembership and also the first \thanks - 
	% these prevent an unwanted space from occurring between the last author name
	% and the end of the author line. i.e., if you had this:
	% 
	% \author{....lastname \thanks{...} \thanks{...} }
	%                     ^------------^------------^----Do not want these spaces!
	%
	% a space would be appended to the last name and could cause every name on that
	% line to be shifted left slightly. This is one of those "LaTeX things". For
	% instance, "A\textbf{} \textbf{}B" will typeset as "A B" not "AB". If you want
	% "AB" then you have to do: "A\textbf{}\textbf{}B"
	% \thanks is no different in this regard, so shield the last } of each \thanks
	% that ends a line with a % and do not let a space in before the next \thanks.
	% Spaces after \IEEEmembership other than the last one are OK (and needed) as
	% you are supposed to have spaces between the names. For what it is worth,
	% this is a minor point as most people would not even notice if the said evil
	% space somehow managed to creep in.
	%
	% The paper headers
	\markboth{Journal of \LaTeX\ Class Files,~Vol.~1, No.~11,~November~2002}{Shell \MakeLowercase{\textit{et al.}}: Bare Demo of IEEEtran.cls for Journals}
	% The only time the second header will appear is for the odd numbered pages
	% after the title page when using the twoside option.
	% 
	% *** Note that you probably will NOT want to include the author's name in ***
	% *** the headers of peer review papers.                                   ***
	
	% If you want to put a publisher's ID mark on the page
	% (can leave text blank if you just want to see how the
	% text height on the first page will be reduced by IEEE)
	%\pubid{0000--0000/00\$00.00~\copyright~2002 IEEE}
	
	% use only for invited papers
	%\specialpapernotice{(Invited Paper)}
	
	% make the title area
	\maketitle
	
	
	\begin{abstract}
		Network reconstruction problem is one of the hot and knotty issues in the research of complex network or network science. In this paper, we use kernel density estimation technique to estimate the distribution of the time difference of arrival in such diffusion process on the basis of stochastic temporal network. We analyze the statistical property discrepancies between edges in the network, then give proof on the left deviation of the estimated survival function on the time-aggregated network. Next we design a probability threshold cutting algorithm which can be used to reconstruct the time-aggregated network of stochastic temporal network. To verify this, we run a lot of simulations on different networks which show high reconstruction speed and accuracy of our algorithm. Last, we discuss the relation between network scale and data amount of the reconstruction procedure which illustrates the compatibility with such large scale network reconstruction problem. Furthermore, a parallelization design idea is presented to speed up the algorithm.
	\end{abstract}
	
	\begin{keywords}
		network reconstruction, stochastic temporal network, time-aggregated network, waiting time distribution,  kernel density estimation.
	\end{keywords}
	% Note that keywords are not normally used for peerreview papers.
	
	% For peer review papers, you can put extra information on the cover
	% page as needed:
	% \begin{center} \bfseries EDICS Category: 3-BBND \end{center}
	%
	% For peerreview papers, inserts a page break and creates the second title.
	% Will be ignored for other modes.
	\IEEEpeerreviewmaketitle
	
	
	
	\section{Introduction}
	% The very first letter is a 2 line initial drop letter followed
	% by the rest of the first word in caps.
	% 
	% form to use if the first word consists of a single letter:
	% \PARstart{A}{demo} file is ....
	% 
	% form to use if you need the single drop letter followed by
	% normal text (unknown if ever used by IEEE):
	% \PARstart{A}{}demo file is ....
	% 
	% Some journals put the first two words in caps:
	% \PARstart{T}{his demo} file is ....
	% 
	% Here we have the typical use of a "T" for an initial drop letter
	% and "HIS" in caps to complete the first word.
	复杂网络,是一种描述现实世界复杂系统的重要工具,在生物、信息、交通等各个领域都有重要的研究意义和实用价值。随着目前信息时代和数据时代的发展,网络科学越来越受到人们的关注。总体来看,网络科学是专门研究复杂网络系统的定性和定量规律的一门交叉科学 ,研究涉及到复杂网络的各种拓扑结构及其性质,比如随机图网络、无标度网络、小世界网络等著名随机网络的研究。与动力学特性 (或功能) 之间相互关系 ,包括动力学同步及其产生机制、网络上的病毒传播及免疫、链路预测、网络可控性研究、网络演化博弈等,以及工程实际所需的网络设计原理及其应用研究,其交叉研究内容十分广泛而丰富。
	
	网络重构问题也是当今网络科学研究中的热点问题和难点问题。在很多情况下,网络的拓扑结构并不能直接被我们观察或测量出来。网络重构就是考虑这样的一种逆问题:通过一些可以观测到的网络动力学行为的数据,来逆向恢复出原始网络的拓扑结构甚至相关性质,也就是所谓的网络重构。我们特别关注了北京师范大学王文旭课题组和复旦大学李翔课题组近年来相关的研究成果。王文旭课题组在近年基于压缩感知[1]技术,在网络重构问题上取得了一系列的成果。首先,他们从进化博弈的角度,利用少量的博弈数据,用压缩感知的方法将网络重构问题转化为稀疏信号的重构问题,而且能在有一定噪声数据的情况下达到很高的重构精度[2]。随后,又从SIS和CP这两种网络传播模型出发,将极度非平凡的传播网络重构问题转化为压缩感知技术框架下的问题,实现了基于二进制时序数据的网络重构[3]。不久后,他们又从个体结点出发,利用Lasso方法,同样将重构问题转化为稀疏信号的恢复问题,将每个结点和其他所有结点之间的连接视为一个稀疏信号的重构,最后整合所有结点的领域信息,从而重构整个网络[4]。Lasso技术中,惩罚项保证了重构的鲁棒性,L1-norm则保证了信号的稀疏性,也就是只需要较少的观测数据。上述基于压缩感知的网络重构都具有一定的鲁棒性。复旦大学李翔课题组近期也在重构时效网络研究取得了一定的突破,考虑了非泊松条件下的传播过程,利用传播过程的到达时间数据实现了重构随机时效网络的有效推断[5]。
	
	网络重构问题虽然取得了一定的发展,但仍然面临着高难度的挑战。如何从不同种类数据中精确恢复完整的网络信息,甚至包括连边方向、权重等,仍然亟待研究。本文研究的内容主要是:以随机时效网络模型为框架,利用扩散过程中的首达时间数据进行随机时效网络模型中的时间累积网络的重构。这一研究的出发点是,在某些情况下,我们并不关心具体的网络交互细节,反而对网络的统计特征更感兴趣,而随机时效网络既包含了底层网络拓扑信息——时间累积网络,又含有点对层面上的时间特性——等待时间分布。相对于传统的基于静态网络模型的重构方法,这种方法更适用于从含有时间特性的数据中恢复出网络拓扑关系。更具体地说,我们的算法只需要知道在扩散过程中,节点的首达时间信息,就可以重构出网络节点之间的连边关系。通过这篇论文的研究,我们可以清晰地刻画随机时效网络模型和其上发生的扩散过程,以及如何利用扩散过程数据的特殊性质进行时间累积网络的重构,阈值剪枝操作和扩散过程数据量又是怎样影响重构精度的。
	
	% You must have at least 2 lines in the paragraph with the drop letter
	% (should never be an issue)

	% needed in second column of first page if using \pubid
	%\pubidadjcol

	% Reminder: the "draftcls" or "draftclsnofoot", not "draft", class option
	% should be used if it is desired that the figures are to be displayed while
	% in draft mode.
	
	% An example of a floating figure using the graphicx package.
	% Note that \label must occur AFTER (or within) \caption.
	% For figures, \caption should occur after the \includegraphics.
	%
	%\begin{figure}
	%\centering
	%\includegraphics[width=2.5in]{myfigure}
	% where an .eps filename suffix will be assumed under latex, 
	% and a .pdf suffix will be assumed for pdflatex
	%\caption{Simulation Results}
	%\label{fig_sim}
	%\end{figure}
	
	
	% An example of a double column floating figure using two subfigures.
	% (The subfigure.sty package must be loaded for this to work.)
	% The subfigure \label commands are set within each subfigure command, the
	% \label for the overall fgure must come after \caption.
	% \hfil must be used as a separator to get equal spacing
	%
	%\begin{figure*}
	%\centerline{\subfigure[Case I]{\includegraphics[width=2.5in]{subfigcase1}
	% where an .eps filename suffix will be assumed under latex, 
	% and a .pdf suffix will be assumed for pdflatex
	%\label{fig_first_case}}
	%\hfil
	%\subfigure[Case II]{\includegraphics[width=2.5in]{subfigcase2}
	% where an .eps filename suffix will be assumed under latex, 
	% and a .pdf suffix will be assumed for pdflatex
	%\label{fig_second_case}}}
	%\caption{Simulation results}
	%\label{fig_sim}
	%\end{figure*}
	
	
	
	% An example of a floating table. Note that, for IEEE style tables, the 
	% \caption command should come BEFORE the table. Table text will default to
	% \footnotesize as IEEE normally uses this smaller font for tables.
	% The \label must come after \caption as always.
	%
	%\begin{table}
	%% increase table row spacing, adjust to taste
	%\renewcommand{\arraystretch}{1.3}
	%\caption{An Example of a Table}
	%\label{table_example}
	%\centering
	%% Some packages, such as MDW tools, offer better commands for making tables
	%% than the plain LaTeX2e tabular which is used here.
	%\begin{tabular}{|c||c|}
	%\hline
	%One & Two\\
	%\hline
	%Three & Four\\
	%\hline
	%\end{tabular}
	%\end{table}
	\section{Diffusion process on stochastic temporal network}
	本章介绍随机时效网络模型,以及其上发生的扩散过程是如何定义及模拟的。
		\subsection{Derivation of Stochastic Temporal Network}
		为了引入随机时效网络,首先我们介绍一般的时效网络$\mathcal{N=(V,E)}$。其中$\mathcal{V}$是时效网络节点的集合,$\mathcal{E}$是节点之间的时效交互事件$\epsilon=(u,v,t,\delta t)$的集合,$(u,v)$表示事件发生的相关节点对,$t\in \left[ 0,T^W \right]$表示事件发生的时间,$T^W$表示观测窗口的长度,$\delta t$表示事件的持续时间,也就表明该事件发生的时间段是$(t,t+\delta t)$。我们假设事件的发生间隔$\delta t$趋近于0。因此,在同一个时间点不可能有两个交互事件同时发生。从而我们把随机时效网络中的事件定义为$(u,v,t)$,抹去了事件的持续时间这一特征,认为时间的发生是在极短的时间内完成的。这个假设主要对应的是信息扩散过程或病毒传播过程。
		
		接着我们构造随机时效网络中的时间累积网络。首先通过映射:$P_{V^2}:V^2 \times \left[0,T^W\right] \mapsto V^2,(u,v,t) \mapsto (u,v)$将所有的时效连边投影成为静态的时间累积网络连边,从而构造出了一个时间累积网络:
		$$\mathcal{G}=P_{V^2}(E)=\left\{ (u,v)|(u,v,t)\in E \right\}$$
		其次,对时间累积网络中的任意一条连边$(u,v)\in \mathcal{G}$,假设其发生事件的事件间隔是服从一个实证分布$\psi(t)$的。具体做法是,记$\left\{ (u,v,t_{uv}^i) \right\}_{i=1,2,\ldots,M}=P_{V^2}^{-1}\left[(u,v)\right]$为连边$(u,v)\in \mathcal{G}$上所有记录下来的交互事件,按照事件的升序排列,$t_{uv}^1<t_{uv}^2<\ldots<t_{uv}^M$。则$\psi_{uv}(t)$是事件间隔时间$\left\{ \Delta t_{uv}^i=t_{uv}^i-t_{uv}^{i-1}\right\}$的实证分布。在数据量有限的情况下,往往可以采用核密度估计的方法来估计出相应的事件间隔分布:
		$$\psi_{uv}(t)=\frac{1}{M}\sum_{i=1}^MK_h(t-\Delta t_{uv}^i)$$
		式中,$M$表示数据个数,$K_h(\cdot{})$表示带宽为$h$的核函数。
		下面我们根据事件间隔分布$\psi_{uv}(t)$引入连边上的等待时间分布$\rho_{uv}(\tau)$。首先,等待时间$\tau_{uv}$定义为在扩散过程中,从节点$u$被首次通知到$(u,v)$上首次出现时效边使得$v$被$u$通知的中继时间。假设不同连边上的等待时间分布是相互独立的,则等待时间$\tau_{uv}$服从一个依据时间间隔分布$\psi_{uv}(t)$进行长度偏差采样到的概率密度分布,写作
		$$\rho_{uv}(\tau)=\frac{1}{m_{uv}} \int_\tau^{\infty}\psi_{uv}(t)dt\Theta(\tau)$$
		其中$m_{uv}=\int_0^\infty t\psi_{uv}(t)dt$表示事件间隔分布$\rho_{uv}(\tau)$的均值,$\Theta(\tau)$表示单位阶跃函数。由此我们可以得到所有时间累积网络上的等待时间分布。
		从上面的步骤中,我们总结推导出了随机时效网络模型下的时间累积网络$\mathcal{G}$和等待时间分布$\rho_{uv}(\tau)$,则随机时效网络可以表示为
		$$\mathcal{N_S=(G,\bm{\rho})},\bm\rho=\left\{ \rho_{uv}(\tau) \right\}_{(u,v)\in \mathcal{G}} $$
		
		接着我们使用生存分析的方法来描述随机时效网络上的扩散过程。首先,给定一个任意的随机时效网络	$\mathcal{N_S=(G,\bm{\rho})}$。我们注意其等待时间分布$\rho_{uv}(\tau)$,规定$\rho_{uv}(\tau)=0,\tau<0$。因为对于扩散过程而言,不存在等待时间$\tau_{uv}<0$的情况。这样一来,就可以写出等待时间分布的规范条件
		$$\int_0^\infty \rho_{uv}(\tau)d\tau=1$$
		在生存分析中,还有一个概念是生存函数(Survival Function),定义为
		$$	\Phi_{uv}(\tau)=1-F_{uv}(\tau)=1-\int_0^\tau \rho_{uv}(\tau)d\tau=\int_\tau^\infty \rho_{uv}(\tau)d\tau$$
		$F_{uv}(\tau)$表示等待时间分布的概率密度函数,$\Phi_{uv}(\tau)$表示连边$(u,v)$在$\tau$时刻之前没有被激活的概率。很显然,对于任意不属于时间累积网络的节点对$(u,v)\notin \mathcal{G}$,$\rho_{uv}(\tau)\equiv 0$。需要特别说明的是,为了问题的规范,我们使用一个支撑集$\left[0,\tau_{max}\right)$来限定等待时间分布。防止由于支撑区间过大,使得难以对其进行核密度估计。
	
		\subsection{Diffusion Process on Stochastic temporal network}
		下面介绍并模拟随机时效网络$\mathcal{N_S=(G,\bm{\rho})}$上发生的扩散过程 。
		简明起见,我们规定时间累积网络$\mathcal{G}$是无向的,即$(u,v)\in \mathcal{G}$和$(v,u)\in \mathcal{G}$的扩散规律是相同的。
		下面考虑扩散过程的发生,算法1给出了该过程的实现。
		\begin{algorithm}
			\caption{Generation of Diffusive Arrival Times}
			\LinesNumbered  
			\KwIn{$\mathcal{N_S=(G,\bm{\rho})}$, $s^*$}
			\KwOut{$\mathcal{D}=\left\{t_v\right\}_{v\in \mathcal{G}}$}
			
			$\left\{w_{uv}\right\}_{(u,v)\in \mathcal{G}} \gets 0$\;	
			\For{each $(u,v)$ in $\mathcal{G}$}{
					$w_{uv} \gets random\_sampling(\rho_{uv}(\tau))$\;  
			}
			$\mathcal{D} \gets dijkstra(\left\{w_{uv} \right\},s^*)$\;
			\Return $\mathcal{D}$\;
		\end{algorithm}	
		通过该算法,我们可以得到一个随机时效网络的一次扩散数据{$\mathcal{D}=\left\{t_v\right\}_{v\in \mathcal{G}}$,如果想得到多个扩散数据,可以从网络中随机选取节点作为扩散源,仿真扩散过程,就可以得到一组扩散过程数据$D=\left\{\mathcal{D}^i\right\}_{i=1,2,...,M}$,其中第$i$次的扩散数据表示为$\mathcal{D}^i=\left\{t_v^i\right\}_{v\in \mathcal{G}}$。$M$代表数据的总组数。
		
		我们考虑在一个固定的随机时效网络上发生的一次信息扩散过程。随机时效网络的时间累积网络无向无自环,等待时间分布服从均质化假设。下面绘制了相应的网络图和扩散图。
		
	
	\section{characteristic differnece analysis of estimated distribution}	
		\subsection{Defination of Edges on Diffusion Process }
		在分析之前,我们特别说明文中提到的几个概念的区别:网络边、非网络边,扩散边和非扩散边。对于随机时效网络 $\mathcal{N_S=(G,\bm{\rho})}$,某一次扩散过程$\mathcal{D}$的扩散路径形成的扩散树为$\mathcal{T}$,如果$(u,v)\in \mathcal{G}$则称$(u,v)$是网络边,如果$(u,v)\notin \mathcal{G}$则称$(u,v)$是非网络边,如果$(u,v)\in \mathcal{T}$则称$(u,v)$是扩散边,如果$(u,v)\in \mathcal{G}$但$(u,v)\notin \mathcal{T}$则称$(u,v)$是网络边则称$(u,v)$是非扩散边。
		
		接着,我们使用$d_{uv}=t_v-t_u$表示由首达时间数据得到节点对的到达时间差。对于扩散边,其首达时间差$d_{uv}$是依据相应的等待时间分布$\rho_{uv}(\tau)$采样得到的,在统计上等于相应的等待时间$d_{uv}=\tau_{uv}$ ;而对于非扩散边或非网络边,其首达时间差$d_{uv}$则是根据网络的拓扑结构和周围节点被通知情况而决定的,不服从等待时间分布$\rho_{uv}$。
		但需要我们注意的是,扩散过程中会普遍存在下述的情况。假设我们知道有这么一条连边$(u^*,v^*)\in \mathcal{G}$存在,当发生了多次信息扩散过程$D=\left\{ \mathcal{D}^i\right\}_{i=1,2,...,M}$时,连边$(u^*,v^*)$受邻居节点的影响,不一定在任意的扩散过程$\mathcal{D}^i$中都是扩散边。
		总之,扩散边和非扩散边是针对于一次扩散过程$\mathcal{D}$而言的。在某一次扩散过程$\mathcal{D}^i$中,不是所有的网络边都会成为扩散边。在这一次传播过程$\mathcal{D}^i$的扩散树$\mathcal{T}^i$中出现的边是扩散边,而且一定属于网络边,因为只有网络连边存在,扩散过程才可能从这条连边上经过,从而成为扩散边;不属于扩散树$\mathcal{T}^i$,但属于网络边的边是非扩散边。因此,在一次扩散中,网络边既可能是扩散边,也可能是非扩散边,此外,非网络边不属于时间累积网络,在任意的扩散过程中都一定不会出现。
		所以,对于假设已知存在的连边$(u^*,v^*)\in \mathcal{G}$作为非扩散边出现在扩散过程中时,会干扰到我们的重构过程。因为作为非扩散边出现的情况下,连边上两个节点的首达时间差$d_{u^*v^*}$同样不服从等待时间分布$\rho_{u^*v^*}(\tau)$的采样,会影响其估计分布的精确度,降低我们认为该边是网络连边的可能性。
		
		\subsection{Distribution Estimation on Edges through Time Differences of Arrival}
		随后,我们用核密度估计的方法进行节点对首达时间差的分布估计。对于上述扩散过程$D$,我们考虑某一连边$(u,v)$上的首达时间差数据$\bm{d}_{uv}=\left\{d_{uv}^i \right\}_{i=1,2,\ldots,M}$的概率分布进行估计,其概率密度函数估计为:
		$$\hat{\rho}_{uv}(\tau)=\frac{1}{M}\sum_{i=1}^MK_h(\tau- d_{uv}^i)$$
		$\hat{\rho}_{uv}(\tau)$表示估计出的概率密度分布。
		最常用的一种核函数是高斯核函数$K(\tau)=\frac{1}{\sqrt{2\pi}}exp(-\tau ^2 /2)$
		高斯核函数的形状和平滑核带宽 将会决定平滑的效果。在我们后续实验中,重构算法并不严重依赖核函数和核带宽的选取,因此核带宽取一个很小的值如0.01,保证数据不会被过渡平滑。
		
		利用KDE方法,我们发现网络边节点对和非网络边节点对的首达时间差估计分布是具有明显差异的。
		\subsubsection{Left-deviation of Estimated Survival Function on True Edges}

		
		\subsubsection{Right-deviation of Estimated Survival Function on False Edges}
		非网络边节点对上的估计分布生存函数相对于网络边节点对上的估计分布生存函数是相对右偏的,即在支撑区间$\left[0,\tau_max\right)$的右侧,$\hat{\Phi}_{(u,v)\in \mathcal{G}}(\tau)> \hat{\Phi}_{(u,v)\notin \mathcal{G}}(\tau)$大概率成立。表现在生存函数的图中,红色估计线总是在蓝色估计线的左下方。且随着横坐标数值的增大,网络边节点对的估计分布生存函数 $\hat{\Phi}_{(u,v)\in \mathcal{G}}(\tau)$(红线)和非连边节点对的估计分布生存函数$\hat{\Phi}_{(u,v)\notin \mathcal{G}}(\tau)$(蓝线)的距离不断增大。这说明,非连边节点对的首达时间差的分布情况由于受到扩散过程在网络拓扑中的随机性的影响,相对于网络边节点对来说更可能分布在区间的右侧,所以其生存函数的降幅才会比网络连边节点对的生存函数的降幅缓慢。
		\subsubsection{Overflow of Estimated Survival Function on False Edges }
		此外,非网络边节点对的估计分布$\hat{\Phi}_{(u,v)\notin \mathcal{G}}(\tau)$在支撑区间$\left[0,\tau_max\right)$右侧的降速更缓慢,且大概率存在数值溢出的情况,现象表现在生存函数图中为$\hat{\Phi}_{(u,v)\notin \mathcal{G}}(\tau)>0$。对于非网络边节点对的首达时间差,由于这个数值不是从等待时间分布采样得到的,不服从等待时间分布,再加上网络拓扑和扩散过程的影响,往往会出现首达时间差超出支撑区间的情况。为了方便统计,我们将首达时间差超出的数值都记录在区间的最右侧,表现在估计分布的图示中,我们可以发现非连边节点对的估计分布的概率密度函数为$\hat{\rho}_{(u,v)\notin \mathcal{G}}(\tau)$(蓝线)在支撑区间的最右端总会出现一个大值,这是所有溢出情况累加的结果;在生存函数中这种现象表现为$\hat{\Phi}_{(u,v)\notin \mathcal{G}}(\tau)>0$ ,而对于而网络边节点对的估计分布$\hat{\rho}_{(u,v)\in \mathcal{G}}(\tau)$、$\hat{\Phi}_{(u,v)\in \mathcal{G}}(\tau)$(红线)和原始分布${\rho}(\tau)$、${\Phi}(\tau)$(紫线)是不存在这种情况的,即$\hat{\rho}_{(u,v)\in \mathcal{G}}(\tau)$、$\hat{\Phi}_{(u,v)\in \mathcal{G}}(\tau)$、${\rho}(\tau)$、${\Phi}(\tau)$ $\simeq 0$
		
	\section{reconstruction algorithm through diffusive arrival times}
	\subsection{Reconsturction Algorithm}
	便于分析,我们在重构算法中使用均质假设,即时间累积网络中所有连边的等待时间分布都是相同的。
	首先,我们定义一个阈值$\theta$和等待时间$\tau_\theta$,分别表示累积分布函数$\Phi_{uv}(\tau)$的上$\theta$分位点处对应的原生存函数概率值和等待时间,即$\Phi_{uv}(\tau_\theta)=\theta$ 。从而,网络重构的问题转化为节点对首达时间差数据的估计分布的分类问题。我们认为当$\theta\to 0$时,对于一个网络边$(u,v)\in \mathcal{G}$,其节点对的首达时间数据出现$d_{uv}>\tau_\theta$的概率$P(d_{uv}>\tau_\theta) \to 0$。而对于整个时间累积网络的重构,只需要把所有可能的节点对 遍历,判断每一个节点对上是否存在连边,最后对所有结果取并,即可得到重构后的时间累积网络 。
	下面我们写出了整个重构算法的流程。
		\begin{algorithm}
		\caption{Reconstruction Algorithm through Diffusive Arrival Times}
		\LinesNumbered  
		\KwIn{$D=\left\{ \mathcal{D}^i\right\}_{i=1,2,...,M},\rho_{uv}(\tau)$}
		\KwOut{$\hat{\mathcal{G}}$}
		$\hat{\mathcal{G}} \gets Empty Graph $\;	
		\For{ $u$ in $\mathcal{V}$}{
			\For{ $v$ in $\mathcal{V}$}{
				$\hat{\rho}_{uv}(\tau)=\frac{1}{M}\sum_{i=1}^MK_h(\tau- d_{uv}^i)$\;
				$\hat{\Phi}_{uv}(\tau)=\int_\tau^\infty \rho_{uv}(\tau)d\tau$\;
				\If{$\hat{\Phi}_{uv}(\tau)>\epsilon$}{
					$\hat{\mathcal{G}} \gets \hat{\mathcal{G}} \cup (u,v)$;
				}
			}
		}
		\Return $\hat{\mathcal{G}}$\;
	\end{algorithm}	

	\subsection{Indicators For Reconstuction Results}
	在我们的算法中,与重构结果最相关的参数是相对数据量 ( 是扩散过程的样本数, 是网络中节点的个数),以及剪枝阈值 的选取。因此,根据数据量 和剪枝阈值 的变化,网络重构性能也会随之变化。怎样合理地设置这两个参数,以及在什么参数条件下重构效果最优,是我们接下来重点讨论的问题。
	为了标准化地衡量重构结果的好坏,我们使用两种标准曲线指标:ROC曲线和PR曲线[48][49],以此衡量重构精度的高低。
	首先我们解释这两种曲线是如何绘制的,以及他们代表着什么样的物理意义。下面我们讨论一个含有正负样本的二分类问题。根据算法分类结果的不同,会出现以下几种情况:TP(True Positive)实际为正样本,被分类为正样本;FP(False Positive)实际为负样本,被分类为正样本;TN(True Negative)实际为负样本,被分类为负样本;FN(False Negative)实际为正样本,被分类为负样本。
	为了不引起歧义,我们使用1表示正样本或分类为正样本,0表示负样本或分类为负样本。在我们使用的ROC曲线或PR曲线中,关注的重点是下面几个指标。
	$$\mathrm{TPR}(\theta)=\mathrm{Recall}(\theta)=\frac{\mathrm{TP}(\theta)}{\mathrm{TP}(\theta)+\mathrm{FN}(\theta)}$$
	$$\mathrm{FPR}(\theta)=\frac{\mathrm{FP}(\theta)}{\mathrm{FP}(\theta)+\mathrm{TN}(\theta)}$$
	$$\mathrm{Precision}(\theta)=\frac{\mathrm{TP}(\theta)}{\mathrm{TP}(\theta)+\mathrm{FP}(\theta)}$$
	对于ROC曲线,随着参数$\theta$的变化,每一个$\theta$都对应着以FPR为横坐标,TPR为纵坐标的二维坐标系上的一个点$\left[TPR(\theta),FPR(\theta)\right]$,所有点组成的轨迹就是该重构结果的ROC曲线。ROC曲线越贴近坐标轴的左上角,表示重构效果越好。同理,对于PR曲线,对应的以Recall为横坐标,Precision为纵坐标的二维坐标系上的点$\left[Recall(\theta),Precision(\theta) \right]$组成的轨迹就是该重构结果的PR曲线。PR曲线越贴近坐标轴的右上角,表示重构效果越好。
	\subsection{Simulations and Results}
	和上一章中的估计分布分析对应,我们同样选取了ER、SF、WS三种随机网络用于算法的验证。此外,额外加入了Football、Lattice2d、Sierpinski三种确定网络作为时间累积网络,以说明算法在一些实际社交关系网络或特殊结构网络上的重构也是有效的。等待时间分布取Gaussian、Uniform、Gumbel三种分布。在上述情况下,我们对随机时效网络上的扩散过程进行了大量仿真,并用仿真得到的扩散过程的首达时间数据进行时间累积网络的重构。在每一张图中,我们都绘制了在相对数据量$C=M/N$分别为0.2、0.4、0.6、0.8、1.0的情况下,随着剪枝阈值$\theta$变化对应的ROC曲线和PR曲线。用F1-Score作为重构效果的最终衡量指标
	$$\mathrm{F1(\theta)}=\frac{2\cdot{\mathrm{Precision}(\theta)}\cdot{\mathrm{Recall}(\theta)}}{\mathrm{Precision}(\theta)+\mathrm{Recall}(\theta)}$$
	我们记录下了在参数$\theta$变化中最优情况下的F1值以及对应的TPR, FPR, Precision, Recall值,绘制成了相应的数据统计表。图表中每一个坐标点或数据都是在10次独立的仿真之后取均值得出的。
	\begin{figure}[t]
		\centering
		\includegraphics[width=3in,height=1.5in]{ERgaussian.eps}
		\includegraphics[width=3in,height=1.5in]{ERuniform.eps}
		\includegraphics[width=3in,height=1.5in]{ERgumbel.eps}
		\caption{ROC and PR Curve of Reconstruction Results}
		\label{1}
	\end{figure}

		
	\begin{table}[tbp]
		\caption{Reconstruction Results Table}
		\begin{center}
			\begin{tabular*}{8cm}{@{\extracolsep{\fill}}lccccc}
				\hline\hline
				& F1 & TPR & FPR & Precision  \\ 
				\hline\hline
				& & Gaussian & &\\
				$C=0.2$ & 0.5335 & 0.8168 & 0.0828 & 0.3961 \\
				$C=0.4$ & 0.8594 & 0.9773 & 0.0197 & 0.7669 \\
				$C=0.6$ & 0.9502 & 0.9628 & 0.0041 & 0.9379 \\
				$C=0.8$ & 0.9864 & 0.9954 & 0.0015 & 0.9775 \\
				$C=1.0$ & 0.9893 & 1      & 0.0014 & 0.9788 \\
				\hline\hline
				& & Uniform & &\\
				$C=0.2$ & 0.4124 & 0.6509 & 0.0921 & 0.3019 \\
				$C=0.4$ & 0.7603 & 0.8253 & 0.0210 & 0.7048 \\
				$C=0.6$ & 0.9224 & 1      & 0.0102 & 0.8560 \\
				$C=0.8$ & 0.9783 & 1      & 0.0027 & 0.9575 \\
				$C=1.0$ & 0.9970 & 1      & 0.0004 & 0.9940 \\
				\hline\hline
				& & Gumbel & &\\
				$C=0.2$ & 0.5336 & 0.7871 & 0.0676 & 0.4270 \\
				$C=0.4$ & 0.9305 & 0.9014 & 0.0173 & 0.7699 \\
				$C=0.6$ & 0.9145 & 0.9488 & 0.0080 & 0.8826 \\
				$C=0.8$ & 0.9441 & 0.9668 & 0.0052 & 0.9225 \\
				$C=1.0$ & 0.9637 & 0.9719 & 0.0029 & 0.9558 \\
				\hline\hline
			\end{tabular*}
		\end{center}
	\end{table}
		
	\section{coclusion}
	The conclusion goes here.
	% if have a single appendix:
	%\appendix[Proof of the Zonklar Equations]
	% or
	%\appendix  % for no appendix heading
	% do not use \section anymore after \appendix, only \section*
	% is possibly needed
	
	% use appendices with more than one appendix
	% then use \section to start each appendix
	% you must declare a \section before using any
	% \subsection or using \label (\appendices by itself
	% starts a section numbered zero.)
	%
	% Use this command to get the appendices' numbers in "A", "B" instead of the
	% default capitalized Roman numerals ("I", "II", etc.).
	% However, the capital letter form may result in awkward subsection numbers
	% (such as "A-A"). Capitalized Roman numerals are the default.
	%\useRomanappendicesfalse
	%
	\appendices
	\section{Proof of the First Zonklar Equation}
	Appendix one text goes here.
	
	% you can choose not to have a title for an appendix
	% if you want by leaving the argument blank
	\section{}
	Appendix two text goes here.
	
	% use section* for acknowledgement
	\section*{Acknowledgment}
	% optional entry into table of contents (if used)
	%\addcontentsline{toc}{section}{Acknowledgment}
	The authors would like to thank...
	
	% trigger a \newpage just before the given reference
	% number - used to balance the columns on the last page
	% adjust value as needed - may need to be readjusted if
	% the document is modified later
	%\IEEEtriggeratref{8}
	% The "triggered" command can be changed if desired:
	%\IEEEtriggercmd{\enlargethispage{-5in}}
	
	% references section
	% NOTE: BibTeX documentation can be easily obtained at:
	% http://www.ctan.org/tex-archive/biblio/bibtex/contrib/doc/
	
	% can use a bibliography generated by BibTeX as a .bbl file
	% standard IEEE bibliography style from:
	% http://www.ctan.org/tex-archive/macros/latex/contrib/supported/IEEEtran/bibtex
	%\bibliographystyle{IEEEtran.bst}
	% argument is your BibTeX string definitions and bibliography database(s)
	%\bibliography{IEEEabrv,../bib/paper}
	%
	% <OR> manually copy in the resultant .bbl file
	% set second argument of \begin to the number of references
	% (used to reserve space for the reference number labels box)
	\begin{thebibliography}{1}
		
		\bibitem{IEEEhowto:kopka}
		H.~Kopka and P.~W. Daly, \emph{A Guide to {\LaTeX}}, 3rd~ed.\hskip 1em plus
		0.5em minus 0.4em\relax Harlow, England: Addison-Wesley, 1999.
		
	\end{thebibliography}
	
	% biography section
	% 
	% If you have an EPS/PDF photo (graphicx package needed) extra braces are
	% needed around the contents of the optional argument to biography to prevent
	% the LaTeX parser from getting confused when it sees the complicated
	% \includegraphics command within an optional argument. (You could create
	% your own custom macro containing the \includegraphics command to make things
	% simpler here.)
	%\begin{biography}[{\includegraphics[width=1in,height=1.25in,clip,keepaspectratio]{mshell}}]{Michael Shell}
	% where an .eps filename suffix will be assumed under latex, and a .pdf suffix
	% will be assumed for pdflatex; or if you just want to reserve a space for
	% a photo:
	
	\begin{biography}{Michael Shell}
		Biography text here.
	\end{biography}
	
	% if you will not have a photo at all:
	\begin{biographynophoto}{John Doe}
		Biography text here.
	\end{biographynophoto}
	
	% insert where needed to balance the two columns on the last page
	%\newpage
	
	\begin{biographynophoto}{Jane Doe}
		Biography text here.
	\end{biographynophoto}
	
	% You can push biographies down or up by placing
	% a \vfill before or after them. The appropriate
	% use of \vfill depends on what kind of text is
	% on the last page and whether or not the columns
	% are being equalized.
	
	%\vfill
	
	% Can be used to pull up biographies so that the bottom of the last one
	% is flush with the other column.
	%\enlargethispage{-5in}
	
	% that's all folks
\end{document}


